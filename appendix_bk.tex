%% -*- coding: utf-8-unix -*-

\chapter{テストシナリオ実装時のトラブルとワークアラウンド}
\label{cpt:troubles}

ネットワークのテストという観点で実際に発見できた問題点については
\ref{sec:statictest-founded-issues}節でとりあげている。本章では、テスト
シナリオ実装の際におきた問題点やワークアラウンドについて解説する。こうし
たワークアラウンドは、事情や理由がわからないと冗長あるいは不合理にみえる
ことがあるため、理由をふくめたノウハウを共有しておくことは重要である。本
章であげた実際のトラブルシュー事例をもとに、テストシナリオのデバッグをど
う考えるかについては\ref{sec:debugging-test-scenario}節で解説している。

本章では特に、テストを繰り返し実行すると成功・失敗がくりかえされてテスト
結果が不定になった例を選定している(\ref{sec:ping-probrem}節・
\ref{sec:telnet-probrem}節)。このようなテストは flaky\footnote{flaky =当
てにならない。} test とも呼ばれ、テストの連続実行(回帰テスト)において問
題となる。また、テストシナリオに本来であれば影響しないと考える条件変化に
よって、テスト成否結果が変化した事例についてもとりあげた
(\ref{sec:SSG-troubles}節)。

 \section{初回のping送受信失敗}
 \label{sec:ping-probrem}

    \paragraph{事象}
L3の通信試験(\code{ping}コマンド)で、最初の1パケット(場合によっては数パ
ケット)だけ送受信に失敗する。そのため、テスト実行ごとにコマンド実行結果
(ping packet loss)判定が変化して、テストが成功したり失敗したりする。

    \paragraph{原因と対処}
テスト対象が物理ネットワークであるため、ARP cache 等その時々のネットワー
ク状態による結果のゆらぎが発生してしまう。よく知られたワークアラウンドと
しては、テスト実行のまえに、予備的なパケット送受信をおこなって、テスト対
象ネットワーク内の状態を更新しておくことである。本プロジェクトでも、
\lstref{lst:ping-workaround}のように ping packet loss 測定前に ping を1
パケット送受信しておく\footnote{本プロジェクトではこれを「捨てping」と呼
んでいる。}ことで回避している~\cite{examples-pr49}。

\begin{lstlisting}[language=,caption=予備的ping実行の例,label=lst:ping-workaround,linebackgroundcolor={\ifnum\value{lstnumber}=3 \color{green!30}\fi}]
When(/^ヨーヨーダイン社内部のユーザ PC に ping$/) do
  cd('.') do
    @src_host.exec "bash -c 'ping #{@user_pc.ip_address} -c 1; exit 0'"
    @src_host.exec "ping #{@user_pc.ip_address} -c 4 > log/ping.log"
  end
end
\end{lstlisting}

 \section{Netcat/telnetコマンドのデータ送受信順序}
 \label{sec:telnet-probrem}

    \paragraph{事象}
Telentを使った通信試験において、\lstref{lst:telnet-bk}に示すようにテスト
が10回に1回程度しか成功しないという現象が発生した。テストではのように、
サーバとして netcat (\code{nc}コマンド)、クライアントとして
\code{telnet} コマンドを使用している。
 
\begin{lstlisting}[caption=Telnetテストシナリオ,label=lst:telnet-bk]
tajima@nettester_2nd:~/repo/examples-tjmtrhs$ bundle exec cucumber features/telnet_internal_network.feature
@static
Feature: 社内テスト環境設定  ヨーヨーダイン社の開発者として、
社内テストサーバにアクセスしたい  なぜならテスト環境設定を行う必要があるから
Scenario: 社内テストサーバへアクセス            # features/telnet_internal_network.feature:8
Given ヨーヨーダイン社内部のテスト環境サーバ        # features/step_definitions/virtual_host.rb:14
 And ヨーヨーダイン社内部のクライアント            # features/step_definitions/virtual_host.rb:6
When 開発者 PC からテストサーバへTelnetでアクセス # features/step_definitions/telnet_steps.rb:2
 Then 社内テストサーバにアクセス成功             # features/step_definitions/telnet_steps.rb:20
expected "Trying 10.10.10.2...\nConnected to 10.10.10.2.\nEscape character is '^]'." to have file content: string includes: "TelnetOK" (RSpec::Expectations::ExpectationNotMetError)
     ./features/step_definitions/telnet_steps.rb:21:in `/^社内テストサーバにアクセス成功$/'      features/telnet_internal_network.feature:12:in `Then 社内テストサーバにアクセス成功' Failing Scenarios: cucumber features/telnet_internal_network.feature:8 #
 Scenario: 社内テストサーバへアクセス 1 scenario (1 failed) 4 steps (1 failed, 3 passed) 0m10.610s
\end{lstlisting}

    \paragraph{調査}

原因を調査するために、同様の操作を手動で実行した場合(成功する場合)とテス
トシナリオ実行で失敗する場合のそれぞれについて strace による調査を実施し
た。\lstref{lst:telnet-success}に成功時、\lstref{lst:telnet-fail}に失敗
時の strace ログを示す。どちらも前半部分は省略してあり、\code{connect}
から 終了(\code{exit})までのログを抜粋してある。

\begin{lstlisting}[language=,caption=Telnet成功時(手動),label=lst:telnet-success,linebackgroundcolor={\ifnum\value{lstnumber}>22 \ifnum\value{lstnumber}<29\color{green!30}\fi\fi}]
socket(PF_INET, SOCK_STREAM, IPPROTO_IP) = 3
setsockopt(3, SOL_IP, IP_TOS, [16], 4)  = 0
connect(3, {sa_family=AF_INET, sin_port=htons(8080), sin_addr=inet_addr("192.168.20.166")}, 16) = 0
open("/etc/telnetrc", O_RDONLY)         = -1 ENOENT (No such file or directory)
open("/home/nwtestsys/.telnetrc", O_RDONLY) = -1 ENOENT (No such file or directory)
write(1, "Trying 192.168.20.166...\nConnect"..., 80Trying 192.168.20.166...
Connected to 192.168.20.166.
Escape character is '^]'.
) = 80
rt_sigprocmask(SIG_BLOCK, NULL, [], 8)  = 0
rt_sigaction(SIGINT, {0x407fe0, [INT], SA_RESTORER|SA_RESTART, 0x7fa39fd524a0}, {SIG_DFL, [], 0}, 8) = 0
rt_sigaction(SIGQUIT, {0x407f90, [QUIT], SA_RESTORER|SA_RESTART, 0x7fa39fd524a0}, {SIG_DFL, [], 0}, 8) = 0
rt_sigaction(SIGWINCH, {0x407f70, [WINCH], SA_RESTORER|SA_RESTART, 0x7fa39fd524a0}, {SIG_DFL, [], 0}, 8) = 0
rt_sigaction(SIGTSTP, {0x40c4b0, [TSTP], SA_RESTORER|SA_RESTART, 0x7fa39fd524a0}, {0x40c4b0, [TSTP], SA_RESTORER|SA_RESTART, 0x7fa39fd524a0}, 8) = 0
ioctl(0, TCGETS, {B38400 opost isig icanon echo ...}) = 0
ioctl(0, SNDCTL_TMR_STOP or TCSETSW, {B38400 opost isig icanon echo ...}) = 0
ioctl(0, TCGETS, {B38400 opost isig icanon echo ...}) = 0
ioctl(0, FIONBIO, [1])                  = 0
ioctl(1, FIONBIO, [1])                  = 0
ioctl(3, FIONBIO, [1])                  = 0
setsockopt(3, SOL_SOCKET, SO_OOBINLINE, [1], 4) = 0
select(4, [0 3], [], [3], {0, 0})       = 1 (in [3], left {0, 0})
recvfrom(3, "TelnetOK\n", 8191, 0, NULL, NULL) = 9
select(4, [0 3], [1], [3], {0, 0})      = 2 (in [3], out [1], left {0, 0})
write(1, "TelnetOK\n", 9TelnetOK
)               = 9
recvfrom(3, "", 8183, 0, NULL, NULL)    = 0
rt_sigaction(SIGTSTP, {SIG_DFL, [TSTP], SA_RESTORER|SA_RESTART, 0x7fa39fd524a0}, {0x40c4b0, [TSTP], SA_RESTORER|SA_RESTART, 0x7fa39fd524a0}, 8) = 0
ioctl(0, TCGETS, {B38400 opost isig icanon echo ...}) = 0
ioctl(0, SNDCTL_TMR_STOP or TCSETSW, {B38400 opost isig icanon echo ...}) = 0
ioctl(0, TCGETS, {B38400 opost isig icanon echo ...}) = 0
ioctl(0, FIONBIO, [0])                  = 0
ioctl(1, FIONBIO, [0])                  = 0
close(3)                                = 0
rt_sigaction(SIGTSTP, {0x40c4b0, [TSTP], SA_RESTORER|SA_RESTART, 0x7fa39fd524a0}, {SIG_DFL, [TSTP], SA_RESTORER|SA_RESTART, 0x7fa39fd524a0}, 8) = 0
ioctl(0, TCGETS, {B38400 opost isig icanon echo ...}) = 0
ioctl(0, SNDCTL_TMR_STOP or TCSETSW, {B38400 opost isig icanon echo ...}) = 0
ioctl(0, TCGETS, {B38400 opost isig icanon echo ...}) = 0
ioctl(0, FIONBIO, [1])                  = 0
ioctl(1, FIONBIO, [1])                  = 0
select(2, NULL, [1], NULL, NULL)        = 1 (out [1])
rt_sigaction(SIGTSTP, {SIG_DFL, [TSTP], SA_RESTORER|SA_RESTART, 0x7fa39fd524a0}, {0x40c4b0, [TSTP], SA_RESTORER|SA_RESTART, 0x7fa39fd524a0}, 8) = 0
ioctl(0, TCGETS, {B38400 opost isig icanon echo ...}) = 0
ioctl(0, SNDCTL_TMR_STOP or TCSETSW, {B38400 opost isig icanon echo ...}) = 0
ioctl(0, TCGETS, {B38400 opost isig icanon echo ...}) = 0
ioctl(0, FIONBIO, [0])                  = 0
ioctl(1, FIONBIO, [0])                  = 0
rt_sigaction(SIGTSTP, {0x40c4b0, [TSTP], SA_RESTORER|SA_RESTART, 0x7fa39fd524a0}, {SIG_DFL, [TSTP], SA_RESTORER|SA_RESTART, 0x7fa39fd524a0}, 8) = 0
ioctl(0, TCGETS, {B38400 opost isig icanon echo ...}) = 0
ioctl(0, SNDCTL_TMR_STOP or TCSETSW, {B38400 opost isig icanon echo ...}) = 0
ioctl(0, TCGETS, {B38400 opost isig icanon echo ...}) = 0
ioctl(0, FIONBIO, [1])                  = 0
ioctl(1, FIONBIO, [1])                  = 0
select(2, NULL, [1], NULL, NULL)        = 1 (out [1])
rt_sigaction(SIGTSTP, {SIG_DFL, [TSTP], SA_RESTORER|SA_RESTART, 0x7fa39fd524a0}, {0x40c4b0, [TSTP], SA_RESTORER|SA_RESTART, 0x7fa39fd524a0}, 8) = 0
ioctl(0, TCGETS, {B38400 opost isig icanon echo ...}) = 0
ioctl(0, SNDCTL_TMR_STOP or TCSETSW, {B38400 opost isig icanon echo ...}) = 0
ioctl(0, TCGETS, {B38400 opost isig icanon echo ...}) = 0
ioctl(0, FIONBIO, [0])                  = 0
ioctl(1, FIONBIO, [0])                  = 0
write(2, "Connection closed by foreign hos"..., 35Connection closed by foreign host.
) = 35
close(-1)                               = -1 EBADF (Bad file descriptor)
exit_group(1)                           = ?
+++ exited with 1 +++
\end{lstlisting}

\begin{lstlisting}[language=,caption=Telnet失敗時,label=lst:telnet-fail,linebackgroundcolor={\ifnum\value{lstnumber}>24 \ifnum\value{lstnumber}<26\color{green!30}\fi\fi}]
socket(PF_INET, SOCK_STREAM, IPPROTO_IP) = 3
setsockopt(3, SOL_IP, IP_TOS, [16], 4)  = 0
connect(3, {sa_family=AF_INET, sin_port=htons(23), sin_addr=inet_addr("10.10.10.2")}, 16) = 0
open("/etc/telnetrc", O_RDONLY)         = -1 ENOENT (No such file or directory)
open("/root/.telnetrc", O_RDONLY)       = -1 ENOENT (No such file or directory)

write(1, "Trying 10.10.10.2...\nConnected t"..., 72Trying 10.10.10.2...
Connected to 10.10.10.2.
Escape character is '^]'.
) = 72
rt_sigprocmask(SIG_BLOCK, NULL, [], 8)  = 0
rt_sigaction(SIGINT, {0x407fe0, [INT], SA_RESTORER|SA_RESTART, 0x7fdd72e1d4a0}, {SIG_DFL, [], 0}, 8) = 0
rt_sigaction(SIGQUIT, {0x407f90, [QUIT], SA_RESTORER|SA_RESTART, 0x7fdd72e1d4a0}, {SIG_DFL, [], 0}, 8) = 0
rt_sigaction(SIGWINCH, {0x407f70, [WINCH], SA_RESTORER|SA_RESTART, 0x7fdd72e1d4a0}, {SIG_DFL, [], 0}, 8) = 0
rt_sigaction(SIGTSTP, {0x40c4b0, [TSTP], SA_RESTORER|SA_RESTART, 0x7fdd72e1d4a0}, {0x40c4b0, [TSTP], SA_RESTORER|SA_RESTART, 0x7fdd72e1d4a0}, 8) = 0
ioctl(0, TCGETS, 0x7ffc1c8e01d0)        = -1 ENOTTY (Inappropriate ioctl for device)
ioctl(0, SNDCTL_TMR_STOP or TCSETSW, {B0 opost isig icanon echo ...}) = -1 ENOTTY (Inappropriate ioctl for device)
ioctl(0, TCGETS, 0x7ffc1c8e01d0)        = -1 ENOTTY (Inappropriate ioctl for device)
ioctl(0, SNDCTL_TMR_START or TCSETS, {B0 opost isig icanon echo ...}) = -1 ENOTTY (Inappropriate ioctl for device)
ioctl(0, FIONBIO, [1])                  = 0
ioctl(1, FIONBIO, [1])                  = 0
ioctl(3, FIONBIO, [1])                  = 0
setsockopt(3, SOL_SOCKET, SO_OOBINLINE, [1], 4) = 0
select(4, [0 3], [3], [3], {0, 0})      = 2 (in [0], out [3], left {0, 0})
sendto(3, "\377\375\3\377\373\30\377\373\37\377\373 \377\373!\377\373\"\377\373'\377\375\5", 24, 0, NULL, 0) = 24
read(0, "", 8191)                       = 0
ioctl(0, TCGETS, 0x7ffc1c8e01a0)        = -1 ENOTTY (Inappropriate ioctl for device)
rt_sigaction(SIGTSTP, {SIG_DFL, [TSTP], SA_RESTORER|SA_RESTART, 0x7fdd72e1d4a0}, {0x40c4b0, [TSTP], SA_RESTORER|SA_RESTART, 0x7fdd72e1d4a0}, 8) = 0
ioctl(0, TCGETS, 0x7ffc1c8e01f0)        = -1 ENOTTY (Inappropriate ioctl for device)
ioctl(0, SNDCTL_TMR_STOP or TCSETSW, {B0 -opost -isig -icanon -echo ...}) = -1 ENOTTY (Inappropriate ioctl for device)
ioctl(0, TCGETS, 0x7ffc1c8e01f0)        = -1 ENOTTY (Inappropriate ioctl for device)
ioctl(0, SNDCTL_TMR_START or TCSETS, {B0 -opost -isig -icanon -echo ...}) = -1 ENOTTY (Inappropriate ioctl for device)
ioctl(0, FIONBIO, [0])                  = 0
ioctl(1, FIONBIO, [0])                  = 0
close(3)                                = 0
rt_sigaction(SIGTSTP, {0x40c4b0, [TSTP], SA_RESTORER|SA_RESTART, 0x7fdd72e1d4a0}, {SIG_DFL, [TSTP], SA_RESTORER|SA_RESTART, 0x7fdd72e1d4a0}, 8) = 0
ioctl(0, TCGETS, 0x7ffc1c8e01c0)        = -1 ENOTTY (Inappropriate ioctl for device)
ioctl(0, SNDCTL_TMR_STOP or TCSETSW, {B0 opost -isig -icanon echo ...}) = -1 ENOTTY (Inappropriate ioctl for device)
ioctl(0, TCGETS, 0x7ffc1c8e01c0)        = -1 ENOTTY (Inappropriate ioctl for device)
ioctl(0, SNDCTL_TMR_START or TCSETS, {B0 opost -isig -icanon echo ...}) = -1 ENOTTY (Inappropriate ioctl for device)
ioctl(0, FIONBIO, [1])                  = 0
ioctl(1, FIONBIO, [1])                  = 0
select(2, NULL, [1], NULL, NULL)        = 1 (out [1])
rt_sigaction(SIGTSTP, {SIG_DFL, [TSTP], SA_RESTORER|SA_RESTART, 0x7fdd72e1d4a0}, {0x40c4b0, [TSTP], SA_RESTORER|SA_RESTART, 0x7fdd72e1d4a0}, 8) = 0
ioctl(0, TCGETS, 0x7ffc1c8e01c0)        = -1 ENOTTY (Inappropriate ioctl for device)
ioctl(0, SNDCTL_TMR_STOP or TCSETSW, {B0 -opost -isig -icanon -echo ...}) = -1 ENOTTY (Inappropriate ioctl for device)
ioctl(0, TCGETS, 0x7ffc1c8e01c0)        = -1 ENOTTY (Inappropriate ioctl for device)
ioctl(0, SNDCTL_TMR_START or TCSETS, {B0 -opost -isig -icanon -echo ...}) = -1 ENOTTY (Inappropriate ioctl for device)
ioctl(0, FIONBIO, [0])                  = 0
ioctl(1, FIONBIO, [0])                  = 0
rt_sigaction(SIGTSTP, {0x40c4b0, [TSTP], SA_RESTORER|SA_RESTART, 0x7fdd72e1d4a0}, {SIG_DFL, [TSTP], SA_RESTORER|SA_RESTART, 0x7fdd72e1d4a0}, 8) = 0
ioctl(0, TCGETS, 0x7ffc1c8e01c0)        = -1 ENOTTY (Inappropriate ioctl for device)
ioctl(0, SNDCTL_TMR_STOP or TCSETSW, {B0 opost -isig -icanon echo ...}) = -1 ENOTTY (Inappropriate ioctl for device)
ioctl(0, TCGETS, 0x7ffc1c8e01c0)        = -1 ENOTTY (Inappropriate ioctl for device)
ioctl(0, SNDCTL_TMR_START or TCSETS, {B0 opost -isig -icanon echo ...}) = -1 ENOTTY (Inappropriate ioctl for device)
ioctl(0, FIONBIO, [1])                  = 0
ioctl(1, FIONBIO, [1])                  = 0
select(2, NULL, [1], NULL, NULL)        = 1 (out [1])
rt_sigaction(SIGTSTP, {SIG_DFL, [TSTP], SA_RESTORER|SA_RESTART, 0x7fdd72e1d4a0}, {0x40c4b0, [TSTP], SA_RESTORER|SA_RESTART, 0x7fdd72e1d4a0}, 8) = 0
ioctl(0, TCGETS, 0x7ffc1c8e01d0)        = -1 ENOTTY (Inappropriate ioctl for device)
ioctl(0, SNDCTL_TMR_STOP or TCSETSW, {B0 -opost -isig -icanon -echo ...}) = -1 ENOTTY (Inappropriate ioctl for device)
ioctl(0, TCGETS, 0x7ffc1c8e01d0)        = -1 ENOTTY (Inappropriate ioctl for device)
ioctl(0, SNDCTL_TMR_START or TCSETS, {B0 -opost -isig -icanon -echo ...}) = -1 ENOTTY (Inappropriate ioctl for device)
ioctl(0, FIONBIO, [0])                  = 0
ioctl(1, FIONBIO, [0])                  = 0
write(2, "Connection closed by foreign hos"..., 35Connection closed by foreign host.
) = 35
close(-1)                               = -1 EBADF (Bad file descriptor)
exit_group(1)                           = ?
+++ exited with 1 +++
\end{lstlisting}

成功時(\lstref{lst:telnet-success})では、\code{connect}後にサーバからの
\verb|recvfrom(3, "TelnetOK\n", ...)|が先にきている(23行目)。しかし、失
敗時(\lstref{lst:telnet-fail})では、クライアントからのネゴシエーションの
\verb|sendto(3, "\377...", ...)|が先に実行されている(25行目)。この場合後
続の\code{write}は実行されていない(そもそも\code{recv}しない)。

これらの strace ログから、送受信順序によるものと仮定した。仮定を実証する
ために、telnetクライアントを改変する。Telnetクライアントが\code{connect}
したあとサーバから来るデータを先に受ける(\code{recv})ために、
\code{connect}のあとに\code{sleep}を入れる
(\lstref{lst:modified-telnet-client})。これをビルドして
(\lstref{lst:howto-build-telnet})、テストステップの中で使用する
(\lstref{lst:telnet-step})。結果として、エラーが発生しなくなることを確認
できた。

\begin{lstlisting}[caption=telnetクライアント改変,label=lst:modified-telnet-client]
--- netkit-telnet-0.17/telnet/netlink.cc        2017-03-21 00:30:45.990808011 +0900
+++ netkit-telnet-0.17-mod/telnet/netlink.cc    2017-03-21 00:24:45.178808011 +0900
@@ -158,6 +158,7 @@
     if (::connect(net, addr->ai_addr, addr->ai_addrlen) < 0) {
        return 1;
     }
+    sleep(10);
     return 2;
 }
\end{lstlisting}

\begin{lstlisting}[language=sh,caption=telnetクライアントのビルド手順,label=lst:howto-build-telnet]
sudo echo "deb http://gb.archive.ubuntu.com/ubuntu/ xenial main restricted" | sudo tee -a /etc/apt/sources.list
sudo echo "deb-src http://gb.archive.ubuntu.com/ubuntu/ xenial main restricted" | sudo tee -a /etc/apt/sources.list
sudo apt-get source telnet
sudo apt-get install libncurses-dev
cd netkit-telnet-0.17
./configure
cd telnet
make
\end{lstlisting}

\begin{lstlisting}[caption=改変telnetクライアントの使用,label=lst:telnet-step]
When(/^ヨーヨーダイン社内部のテスト環境サーバに telnet でログイン$/) do
  cd('.') do
    @telnet_service = AsyncExecutor.new(host: @test_host, result_file: 'log/telnet_server.log')
    @telnet_service.exec "bash -c 'echo LoginOK | sudo nc -l 23'"
    @src_host.exec "bash -c '/home/nwtestsys/examples/telnet-source/netkit-telnet-0.17/telnet/telnet #{@test_host.ip_address} > log/login.log; exit 0'"
  end
end
\end{lstlisting}

    \paragraph{結果と対処}
サーバ (netcat) はクライアントからの接続を受け付けて echo するとすぐに切
断する。結果としてサーバから送られたデータの \code{recv} とクライアント
が送るデータの \code{sendto} のどちらが早いかという順序(タイミング)問題
が発生していたと考えられる。

こうした問題はサーバ側に telnet デーモン (telnetd) を使用することで回避
可能である。しかし、telnetd を使用すると対話的処理が必要になってしまうた
め、テストステップが複雑になる。そこで、クライアント側に Net::Telnet を
使用することで問題を回避している(\lstref{lst:telnet-step-new})。

\begin{lstlisting}[caption=修正後テストステップ,label=lst:telnet-step-new,linebackgroundcolor={\ifnum\value{lstnumber}=5 \color{green!30}\fi}]
When(/^社内のテスト環境サーバに telnet でログイン$/) do
  cd('.') do
    @telnet_service = AsyncExecutor.new(host: @test_host, result_file: 'log/telnet_host.log')
    @telnet_service.exec "bash -c 'echo LoginOK | sudo nc -l 23'"
    @src_host.exec "ruby -e \"require 'net/telnet'; Net::Telnet.new('Host' => '#{@test_host.ip_address}', 'Port' => '23').cmd('') {|res| print res}\" > log/login.log"
  end
end
\end{lstlisting}

 \section{Juniper SSG の不正なパケット処理に関連するトラブル}
 \label{sec:SSG-troubles}

この節では、
\begin{itemize}
 \item NetTesterサーバまたは物理OpenFlowスイッチによる不正なパケットの生成
 \item FW (Juniper SSG)による不正なパケットの処理
\end{itemize}
に関連すると考えられた事象について解説する。いずれもワークアラウンドによっ
て回避しており、詳細原因までは追求できていないが、テスター・テスト対象ネッ
トワーク(内部の機器)それぞれの動作によるトラブルだと推測される。こうした
組み合せによるトラブルでは不可解な事象が発生し、かつ原因の推定が難しいた
め注意が必要である。

 \subsection{TCP checksum offload 動作に関連するトラブル}
 \label{sec:tcp-checksum-offload-trouble}

\ref{sec:nettester-server-software}節では、トラブルによりテェックサム計
算オフロードを無効化を設定していることを述べた。本節ではその際の事象と調
査内容について解説する。

    \paragraph{事象}
テストシナリオ中、TCP接続をおこなうツールを利用したとき、通信が成功せず
にテストが失敗する。あるいはTCPコネクションが確立するが、単純な接続確認・
送受信しかしないテストにもかかわらず、終了するまでに2分前後かかる。TCPで
はなくICMP(ping)に変更すると、同様のテスト条件でも問題が発生しない。

    \paragraph{調査}
通信種別によって問題の有無が変化するため以下のように順を追って調査している。
\begin{itemize}
 \item テストノードのパッチ接続先の条件による事象変化の確認。(パッチ接続
       先ポートの access/trunk による違いの確認。)
 \item Tester set の変更による事象変化の確認。(異なる物理OFS/サーバによ
       るハードウェア等の切り分け。)
 \item テストトラフィックのパケットキャプチャ取得と比較。(テストノード
       /NetTesterサーバ出口/OFS入出力/テスト用ネットワークにあるL2SW/FW
       それぞれの入出力などを個別にキャプチャ取得。)
\end{itemize}

最終的に、以下の条件で TCP パケットのチェックサムが incorrect となってい
ることがわかった。
\begin{itemize}
 \item テスト用ノードをテスト対象ネットワークに trunk port で接続させる。
       (OVS で VLAN Tag を操作する。)
 \item テストトラフィックとして TCP 通信をおこなう。
       \begin{itemize}
        \item SYN/FIN フラグがついたトラフィックについては問題が発生しな
              い。(TCPで特定のフラグがついたパケットで問題がおきる。)
        \item ICMP 通信については問題が発生しない。
       \end{itemize}
 \item L2SW(Cisco Catalyst3750G)内のL2セグメントで折りかえす場合は問題発
       生せず。FW (Juniper SSG)を経由する通信(テストトラフィック)につい
       て影響をうけている。
\end{itemize}

条件を変えながらキャプチャをみていると、VLAN tag ありでテスト対象ネット
ワークへテストノードをパッチするケースでは、TCP sequence number が不正に
なったり、その結果としてパケット再送が発生していることがわかった。テスト
ノードを access port へパッチすると発生しないことから、NetTester による
VLAN tag 操作がおこなわれるポイントでの問題と考えられた。

実際にどのようなパケットが生成されていたかを\lstref{lst:capture-untag}お
よび\lstref{lst:capture-tag}に示す。それぞれ、NetTesterサーバ(KVMホスト
上の仮想マシン)のSSW--PSW間通信をパススルーしているKVMホストの物理NICで
取得したものである(\figref{fig:poc-env-physical-detail}のem2)。VLAN Tag
およびチェックサムについてまとめると\tabref{tab:capture-result}のように
なる。

\begin{lstlisting}[language=,caption=KVM Hostキャプチャ(タグなし),label=lst:capture-untag]
20:59:31.948164 00:d9:87:40:c2:7d > 00:10:db:ff:20:60, ethertype IPv4 (0x0800), length 66: (tos 0x0, ttl 64, id 1192, offset 0, flags [DF], proto TCP (6), length 52)	10.10.10.4.44910 > 10.10.0.100.80: Flags [.], cksum 0x1ea2 (incorrect -> 0xbd9a), seq 1, ack 1, win 229, options [nop,nop,TS val 3228510328 ecr 3228510328], length 0
20:59:31.948196 00:d9:87:40:c2:7d > 00:10:db:ff:20:60, ethertype 802.1Q (0x8100), length 70: vlan 2025, p 0, ethertype IPv4, (tos 0x0, ttl 64, id 1192, offset 0, flags [DF], proto TCP (6), length 52)	10.10.10.4.44910 > 10.10.0.100.80: Flags [.], cksum 0xbd9a (correct), seq 1, ack 1, win 229, options [nop,nop,TS val 3228510328 ecr 3228510328], length 0
\end{lstlisting}

\begin{lstlisting}[language=,caption=KVM Hostキャプチャ(タグあり),label=lst:capture-tag]
20:59:33.565558 00:2a:28:4d:69:2f > 00:10:db:ff:20:60, ethertype 802.1Q (0x8100), length 70: vlan 2023, p 0, ethertype IPv4, (tos 0x0, ttl 64, id 57905, offset 0, flags [DF], proto TCP (6), length 52)	10.10.0.100.80 > 10.10.10.4.44910: Flags [.], cksum 0x1ea2 (incorrect -> 0xba70), seq 1, ack 3, win 227, options [nop,nop,TS val 3228510733 ecr 3228510733], length 0 
20:59:33.565738 00:10:db:ff:20:60 > 00:d9:87:40:c2:7d, ethertype IPv4 (0x0800), length 66: (tos 0x0, ttl 63, id 57905, offset 0, flags [DF], proto TCP (6), length 52)	10.10.0.100.80 > 10.10.10.4.44910: Flags [.], cksum 0x1ea2 (incorrect -> 0xca49), seq 4294963240, ack 3, win 227, options [nop,nop,TS val 3228510733 ecr 3228510733], length 0
\end{lstlisting}

\begin{table}[h]
 \centering
 \caption{パケットキャプチャ結果}
 \label{tab:capture-result}
 \begin{tabularx}{\linewidth}{l|X|X}
  \hline
  パッチのVLAN Tag設定
  & NetTester Server to Testee (1行目)
  & Testee to NetTester Server (2行目,L3終端後) \\
  \hline
  \hline
  Tag なし (\lstref{lst:capture-untag})
  & Checksum incorrect
  & Checksum correct \\
  Tag あり (\lstref{lst:capture-tag})
  & Checksum incorrect
  & Checksum incorrect \\
  \hline
 \end{tabularx}
\end{table}

    \paragraph{結果と対処}
\tabref{tab:capture-result}に示したように、VLAN tag などいくつかの条件で
はチェックサムが incorrect になってしまい、テストトラフィックとして生成
したTCP通信がうまくいかなかった。\tabref{tab:capture-result}の結果からサー
バ物理NICでのパケット処理機能がうまく機能していないと予想された。そこで、
対処としてチェックサムオフロードの機能を無効~\cite{net-tester-pr7}にして
いる(\ref{sec:nettester-server-software}節)。
\begin{lstlisting}[language=,]
sudo ip netns exec [test node (netns)] ethtool -K [test node veth] tx off
\end{lstlisting}

根本的な原因までは調査・追求していない。Tag (dot1q tag) がある場合にチェッ
クサムオフロードがうまく機能しない理由については不明(未調査)である。原因
としては、FW(SSG)が不正なチェックサムをもつパケットをうまく処理できずに、
一部だけ通してしまう可能性も考えられる。

参考情報として、NICの情報を\lstref{lst:nic-info}に、NICドライバの情報を
\lstref{lst:nic-driver-info}に示す。

\begin{lstlisting}[language=,caption=NIC情報,label=lst:nic-info]
tajima@NW-TEST-SYS-PJ-SERV:~$ sudo lspci | grep -i network
01:00.0 Ethernet controller: Intel Corporation I350 Gigabit Network Connection (rev 01)
01:00.1 Ethernet controller: Intel Corporation I350 Gigabit Network Connection (rev 01)
07:00.0 Ethernet controller: Intel Corporation I350 Gigabit Network Connection (rev 01)
07:00.1 Ethernet controller: Intel Corporation I350 Gigabit Network Connection (rev 01)
\end{lstlisting}

\begin{lstlisting}[language=,caption=NICドライバ情報,label=lst:nic-driver-info]
tajima@NW-TEST-SYS-PJ-SERV:~$ sudo ethtool -i em2
driver: igb
version: 5.2.13-k
firmware-version: 1.61, 0x8000090e
bus-info: 0000:07:00.1
supports-statistics: yes
supports-test: yes
supports-eeprom-access: yes
supports-register-dump: yes
supports-priv-flags: no
\end{lstlisting}

 \subsection{物理OFSリプレースにともなうFW動作トラブル}
 \label{sec:ofs-replace-fw-trouble}

    \paragraph{事象}
本プロジェクトの検証環境は沖縄オープンラボラトリのテストベッドを借用した。
OpenFlowスイッチなどもオープンラボの機材を利用している。一部のOFS(OFS1)
についてオープンラボ内部の機材交換が発生した際に、交換前はすべて成功して
いたテストの一部が失敗するようになった。このとき、テストをくりかえすと成
功・失敗は変化し、常に失敗し続けるわけではない(高頻度で失敗するシナリオ
がある)。

    \paragraph{調査}
失敗するテストシナリオを比較すると、L3でセグメントを経由する通信をおこな
うテストシナリオが高頻度で失敗することがわかった。PoC環境のL3 (default
gateway)はFWであるため、FWの動作に何らかの問題があることが想定された。問
題箇所のしぼりこみ・切り分け対応としては以下の調査をおこなった。

\begin{itemize}
 \item テスト用ノード(クライアント・サーバ)側でパケットキャプチャ
       \begin{itemize}
        \item FWでNATしている(セッション情報が表示される;
              \lstref{lst:fw-nat-info})ものの、実際にテストノード側でパ
              ケットを見るとNATされていない(\lstref{lst:server-capture})。
        \item テスト用ノードからreplyしていても届いていないことがある。
       \end{itemize}
 \item FW1/2 を再起動させる : 問題解決せず
 \item 物理OFSのフローテーブルのクリア : 問題解決せず
 \item L2SW, SSGの MACアドレステーブルのチェック
       \begin{itemize}
        \item SSGがサーバ側のMACアドレスを学習できていない。
        \item テストノードから default gateway への ICMP に応答がない。
       \end{itemize}
 \item FWハードウェアのきりわけ
       \begin{itemize}
        \item FW1のアップリンクをリンクダウンさせ、強制的にFW2がactiveに
              なるように変更してテストを実行。テストは通るが、いくつかの
              テストシナリオでは完了までに時間がかかる(数分)。
       \end{itemize}
\end{itemize}

\begin{lstlisting}[language=,caption=FW NATセッション情報,label=lst:fw-nat-info]
test:FW1(M)-> get session
nat used ipv6 addr: allocated 0/maximum 64256
alloc 4/max 16064, alloc failed 0, mcast alloc 0, di alloc failed 0
total reserved 0, free sessions in shared pool 16060
id 15689/s**,vsys 0,flag 48000000/4000/0001/0000,policy 24,time 2, dip 4 module 0
 if 6(nspflag 801805):10.10.0.100/45402->198.51.100.3/80,6,00217094feda,sess token 18,vlan 2023,tun 0,vsd 0,route 5,wsf 0
 if 0(nspflag 10800804):203.0.113.6/8637<-198.51.100.3/80,6,00211c2c7245,sess token 17,vlan 2021,tun 0,vsd 0,route 7,wsf 7
\end{lstlisting}

\begin{lstlisting}[language=,caption=テストノード(サーバ)側キャプチャ,label=lst:server-capture]
tajima@nettester_1st:~$ sudo tcpdump -i ens5 -n -nn -v
tcpdump: listening on ens5, link-type EN10MB (Ethernet), capture size 262144 bytes
06:34:13.324210 ARP, Ethernet (len 6), IPv4 (len 4), Request who-has 10.10.0.1 tell 10.10.0.100, length 28
06:34:13.326263 ARP, Ethernet (len 6), IPv4 (len 4), Request who-has 10.10.0.1 tell 10.10.0.100, length 46
06:34:13.332088 ARP, Ethernet (len 6), IPv4 (len 4), Reply 10.10.0.1 is-at 00:10:db:ff:20:60, length 46
06:34:13.332542 IP (tos 0x0, ttl 64, id 32529, offset 0, flags [DF], proto TCP (6), length 60)
    10.10.0.100.45402 > 198.51.100.3.80: Flags [S], cksum 0xc51c (correct), seq 2276330937, win 29200, options [mss 1460,sackOK,TS val 578056872 ecr 0,nop,wscale 7], length 0
06:34:13.339657 IP (tos 0x0, ttl 64, id 32529, offset 0, flags [DF], proto TCP (6), length 60)
    10.10.0.100.45402 > 198.51.100.3.80: Flags [S], cksum 0xc51c (correct), seq 2276330937, win 29200, options [mss 1460,sackOK,TS val 578056872 ecr 0,nop,wscale 7], length 0
\end{lstlisting}

    \paragraph{結果と対処}

最終的には、交換(リプレース)した物理OpenFlowスイッチ(OFS1)の再起動で解決
した。原因としては次のような要因が推測される(詳細な原因追求はできていな
い)。
\begin{itemize}
 \item \ref{sec:tcp-checksum-offload-trouble}節のように、FW(SSG)はパケッ
       トに不備があっても通してしまう事例がある。この事例でも物理
       OFS(OFS1)に何らかの不具合があって、パケットが全部は送信されていな
       いと仮定する。L2SWEX--FW1--L2SW1間は障害試験のためにOFS1を経由す
       る(\figref{fig:patch-layer1},
       \figref{fig:poc-env-physical-detail})。FW1がactiveな場合、L2SW-FW
       間トラフィックがOFS1を通るのでパケットに問題が発生する。しかしSSG
       はそれを無視して通す。FW2をactiveにするとOFS1を経由しないためにテ
       ストが成功する。
 \item 物理OpenFlowスイッチを交換したとき、交換後のスイッチは
       \code{fail\_mode=secure}ではなかったため learning switch として動
       作していた(\ref{sec:nettester-deploy-psw}節)。そのため、L2SWでは
       VLAN mismatch となっていた。その後\code{fail\_mode}を設定して
       VLAN mismatch 解決された。L2SW 側でインタフェース状態を確認してい
       たものの、物理OFS--L2SW間ポートに何かの状態が残っていた可能性があ
       る。(OFS1の再起動でおきたポートの up/down により問題が解決した。)
\end{itemize}

%%% Local Variables:
%%% mode: yatex
%%% TeX-master: "main.tex"
%%% End:
