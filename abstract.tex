%% -*- coding: utf-8-unix -*-

\chapter{はじめに}
\label{chap:abstract}

 \section{本書の目的}
 \label{sec:book-purpose}

本書は、沖縄オープンラボラトリ~\cite{ool-web}で2016年度に実施された「ネッ
トワークテストシステムプロジェクト」の活動報告書である。本書の目的は、プ
ロジェクトの目標と実施した活動、活動の結果・成果をまとめ、報告することで
ある。特に、最終的な成果(物)についての解説だけでなく、そこへいたるまでの
検討過程・判断や選択の基準などについてもとりあげ、「考えかた」についても
共有することを目的としている。

 \section{本プロジェクトの目的}
 \label{sec:pj-purpose}

% TODO: OOD発表資料のp1 : 視点/観点の話をいれること
% OOD発表資料のp.2-3

情報システムを構築し、利用者に提供していくにあたって、システムを構築・運
用する側は、様々な製品や技術をくみあわせてシステムを構成し、サービスを実
現していく必要がある。複数の仮想化技術の導入、拡張性に対する要求とそれを
実現するための自動化技術の発展などにより、システムは抽象化されてより柔軟
な制御が可能になった。その反面、大規模かつ複雑にもなっている。こうした、
多数のブラックボックスをくみあわせた複雑なシステムの構築や運用では、シス
テム構成要素間の連携を把握し、システム内部の一部の変更によってシステム全
体の動作やサービスにどういった影響があるかを判断することは難しくなる。一
方で、利用者の要求やサービスの変化のサイクルは速くなっており、システム開
発者あるいはサービス提供者は、その要求に追従していくことが求められる。

このような状況から、今後サービス提供者は、サービス全体の動作を判断し、最
終的に利用者に対して価値が提供できているかどうか、サービス利用者の観点で
システムの状況を判断することがより重要になっていく。本プロジェクトは、サー
ビス変化の迅速性を考えるうえでボトルネックになりがちなネットワークに着目
する。より高速かつ網羅的にネットワークが提供するサービスの正常性を確認す
る(テストする)ためのシステム = ネットワークテストシステムの提案、および
その有効性を確認するための実証実験をおこなう。これによって従来のネットワー
ク構築・運用の迅速性の向上、リスク低減をおこない、CI/CDプロセスを実現さ
せることを目指す。

 \section{関連研究・関連プロダクト}
 \label{sec:related-research}

沖縄オープンラボラトリでは、本プロジェクト以外にネットワークテストに関す
る研究が複数ある。OF-Patchプロジェクト~\cite{ool-testbedpj}はOpenFlow制
御による物理パッチパネルを実装・OSS公開している。また、それによって実際
に沖縄オープンラボのテストベッドを構築し、利用者に提供している。VNFテス
トシナリオ自動化PJ~\cite{ool-vnftestpj}は複数のネットワーク仮想アプライ
アンスのデプロイパターンや相互接続性に着目しており、ハイパーバイザとVNF,
VNF同士の接続などさまざまなパターンで接続性や性能測定を網羅的に自動実行
するシステムを構築している。OF-PatchはL1(物理トポロジ)操作に、VNFテスト
シナリオ自動化はVNFの利用(組合せ・機能・特性)に焦点をあてている。

End-to-endの通信やネットワーク全体のふるまい、物理ネットワークのテストに
着目したテストや検証の自動化に関する関連研究として、ホワイトボックススイッ
チと従来の(レガシーな)ネットワーク機器で構成されたテストベッドネットワー
クに対して、OSSを組み合わせて任意の箇所に自動で障害を発生させる検証自動
化システム~\cite{wbsw-oss-test-automation}が提案されている。レガシー機器
の利用・end-to-endでのふるまいへの着目など本研究と課題感が近く、テスト環
境の自動構築などテストとして求められる機能一式に対応している。本研究では
テスト対象環境の自動構築には比重を置いていないこと、テストトラフィックの
ディストリビューションにOpenFlow制御を利用していること、テストシナリオ実
装の考えかたなどに違いがある。

商用製品としては、FWのフィルタポリシテストの自動化製品がある
(NeedleWork~\cite{needlework-web})。NeedleWorkは小型のアプライアンスとし
て実装されており、FW単体(あるいはFWに接続された3ゾーン(セグメント)構成の
ネットワーク)をテスト対象としている。また、内部ではnetcatを使用してお
り~\cite{needlework-slide}、L4レベルのテストトラフィック生成をおこなうこ
とができる。

ネットワークのテストという観点で利用可能なOSSとしては、
InfraTester~\cite{infratester-github}、ToDD~\cite{todd-github}、
OpenVNet~\cite{openvnet-web}がある。InfraTesterはネットワークを経由して、
特定のサービスの外側からの動作(ふるまい)をテストすることができる。構成と
して、定点(InfraTesterサーバ)から見たときのサービスを観測するかたちにな
るため、L2/L3のネットワーク(物理的な分散性)についてはターゲットとしてい
ない。ToDDは、テスト対象ネットワークにAgentを配布し、ToDDサーバとメッセー
ジキューによって情報交換をおこなう~\cite{todd-blog}。テスト自体はtestlet
という形でToDDサーバからAgentへおくられ、実行される。Agentを使うことで、
ネットワークの物理的な分散性(distribution)に対応しており、拡張性のたかい
アーキテクチャをもっている。OpenVNetはOpenFlowを利用したネットワーク仮想
化システムだが、本プロジェクトと同様の課題感から物理ネットワークのテスト
への対応をすすめている。OpenVNetで構成した仮想ネットワーク(オーバーレイ)
に対してOpenFlow対応ハードウェアを介して物理ネットワーク(ネットワーク機
器)を接続する~\cite{openvnet-slide,network-testing-sdn-atmarkit}。仮想ネッ
トワーク(論理構成)についてはterraformやcloudifyなどのオーケストレーショ
ンツールを併用し、論理構成を自由に記述し、再現・再構成可能とすることで検
証環境構成の自由度を確保している。

本PoCではターゲットとしていない(\ref{sec:discuss-network-test}節参照)が、
テスト対象となるネットワーク機器の操作をおこなうための仕組みに関する近年
の動向について簡単に触れておく。従来のネットワーク機器API(CLI)を抽象化す
るためのOSS開発~\cite{netmiko-github,napalm-github}は活発におこなわれてい
る\footnote{本研究では相当する機能について、よりシンプルで簡易的なツール
を使用している(\ref{sec:expectacle}節参照)。}。また、
Ansible~\cite{ansible-web}はバージョン2.2よりネットワーク機器の操作に対
応~\cite{ansible-22-news}しており、ネットワークの構築・運用自動化分野での
利用事例が増えていくことが予想される。技術動向としては、Google等のネット
ワークオペレータによってベンダ中立なネットワーク機器インタフェース(API)
策定の活動がおこなわれている(OpenConfig~\cite{openconfig})。OpenConfigに
ついてはJuniper, Cisco, Aristaなどの商用製品による採
用~\cite{openconfig-news}もはじまっている。

 \section{前提事項}
 \label{sec:premise}

本書の読者として、ネットワークテストシステムの有効性の判断や、利用・導入
を検討をおこないたいエンジニアを想定している。以下にあげる点については基
礎知識として本書ではとりあげない。

\begin{itemize}
 \item Linuxの基本的な使用方法
       \begin{itemize}
        \item KVM による仮想マシン(VM)の操作/管理
        \item 必要なツールやソフトウェアのインストール/パッケージ管理
        \item git の使用方法
        \item Linux Network Namespace および iproute2 ( \code{ip} コマン
              ド)によるNamespace の操作
       \end{itemize}
 \item 基本的なネットワークに関する知識
       \begin{itemize}
        \item TCP/IP および Ethernet/VLAN
        \item Cisco や Juniper の L2/L3 Switch, Firewall の基本的な動作・
              設定
       \end{itemize}
 \item OpenFlow
       \begin{itemize}
        \item OpenFlow/1.0
        \item Trema による OpenFlow Controller の実装
        \item OpenFlow スイッチ (Open vSwitch) の基本的な動作・設定
       \end{itemize}
 \item Rubyプログラミング
       \begin{itemize}
        \item RubyEnv をつかった ruby 環境の管理
        \item Bundler をつかったパッケージ管理
        \item Cucumber によるテストシナリオ実装
       \end{itemize}
\end{itemize}

また、本プロジェクトの前身となる、2015年度にOOLで実行したプロジェクト
(「L1patch 応用ネットワークテストシステム」プロジェク
ト~\cite{ool-l1pj-web}、通称\lopj)でとりあつかった内容については原則とり
あげない。2015年度プロジェクトについては、\lopjpoc や \lopjtech を参照す
ること。

%%% Local Variables:
%%% mode: yatex
%%% TeX-master: "main.tex"
%%% End:
