%% -*- coding: utf-8-unix -*-

\chapter{課題設定}
\label{chap:problem-setting}

 \section{ネットワークのテストにおける課題}
 \label{sec:nw-test-problem}

% OOD発表資料のp.2-3
% なぜ「ネットワークのテスト」を対象とするのか?
% ネットワークのテストの何が難しいのか?
% これまでネットワークのテストとしてどういったことをおこなっていたのか?

基本的な課題設定については L1patch プロジェクト試験結果レポー
ト~\cite{l1pjpoc} を参照すること。ここでは簡単に解説する。

  \subsection{なぜネットワークのテストに注目するのか}
  \label{sec:reason-to-focus-network}

\ref{sec:pj-purpose}節で示したとおり、本書では情報システムの構成要素とし
ての「ネットワーク」を対象としている。ネットワークに着目する主な理由は、
以下の2点である。
\begin{itemize}
 \item まだ自動化できないことが多い
 \item サービス展開のボトルネックになりがち
\end{itemize}
\ref{sec:difficulty}および\ref{sec:bottleneck}節でこの2点について解説する。

  \subsection{自動化の難しさ}
  \label{sec:difficulty}

ネットワークで自動化がすすまない理由はいくつかあるが、ここではテストの自
動化という観点から、主要な課題について解説する。

    \paragraph{垂直統合の歴史}
歴史的に、ネットワーク機器はベンダごとに異なるOS/API(コマンド)をもち、共
通のインタフェースが存在しない。そのため、異なるベンダの機器をつかったネッ
トワークを作ろうとした場合、設定としてはおなじ操作であっても、異なる
API(コマンド)で操作する必要がある。ネットワークに対する操作の自動化はこ
れまでもおこなわれているが、機器(OS)ごと、機器の設計上の役割や運用上のオ
ペレーションごとに多数の自動化スクリプトを用意する必要があり、複雑かつ汎
用性が低い状態になっている。また、複数のデバイスを操作するうえでは、設定
が反映され動作がきりかわるタイミングなどをふまえたうえで、全体のワークフ
ローなどを考える必要があるといった課題もある。そのため自動化されるのは、
シンプルで定型的な処理にとどまることが多い。\footnote{NW機器を抽象化し統
一した方法で異なるOS/APIの機器を操作可能にする製品やOSSプロダクトなども
存在する。しかし、対応していない製品の利用にあたっては「ドライバ」とよば
れる操作対象機器のAPIや取得情報などを別途開発する必要があるなど、コスト
がかかる。APIについては、Netconf/YANGなどをベースにしたインタフェースや
データモデル標準化の動きはあるものの、現時点では実装されている機器はまだ
少数であり、ベンダ/OSごとに個別にとりあつかう必要がある、という状況であ
る。}

    \paragraph{物理的な位置の操作}
ネットワーク自体が情報システムの構成要素(計算機リソースなど)の物理的な配
置を抽象化する機能をもつため、ネットワークそれ自体のテストについては、物
理的な場所を考慮する必要がある。こうした物理構成上の要求が発生するテスト
\footnote{例えば、リンクダウンなどの物理障害を発生させるケース、ネットワー
ク機器の追加(拡張)・削除といったネットワークの物理構成(トポロジ)を変更す
るケースなど。} は、その「実体を直接操作したい」という要求の性質上、自動
化することが難しい。

    \paragraph{テストケースの組み合わせ爆発}
ネットワークは自律分散制御され、機器相互での通信規約の整合性をとることで、
end-to-end の通信が実現される。ネットワークが狙ったとおりに動作している
かどうか、というテストでは、物理構成・論理構成を加味した多数の組み合わせ
を考慮する必要がある。ネットワークのテストパターンは、ネットワークを構成
している機能要素の組み合わせによって決まるため、構成要素の増加にともない
爆発的に増加してゆく。特に近年では仮想化技術の導入がすすみ、テストパター
ンもより多くなる傾向がある。

  \subsection{サービス展開のボトルネック}
  \label{sec:bottleneck}

情報システムの構成要素はネットワークを経由して接続される。そのため、特に
ネットワークの上流では複数の利用者・システム・顧客(テナント)がひとつのネッ
トワークを共有して利用している。共有度の高いコンポーネントでの障害は、関
連するより広い範囲へ影響するため、より慎重な設計やオペレーションが求めら
れる。また、ネットワークは十分な拡張性・冗長性をもつ必要があるため、自律
分散制御をとるかたちで構成される。ネットワーク機器は相互に制御情報(シグ
ナル)を交換し、周囲の状況にあわせて自律的にトラフィックを処理することで、
拡張性・冗長性を保証していく。

こうした理由により、システム内の一部で発生したイベントがおよぼす影響度や
範囲を完全に予測することは難しい。

特にクラウドサービス(IaaS)のようなシステムを運用している場合、ネットワー
ク上にある個々のシステムでの操作は利用者側で自動化され、構築や運用上の変
更がすばやくできるようになっていたとしても、それにあわせてサービス提供者
側のネットワークが追従できないことがある。また、サービス提供者として必要
なネットワーク側(利用者に提供するネットワークリソース)の拡張・メンテナン
ス(OS更新など)の際に、利用者へのサービス影響がでないかどうかを確認するた
めの作業にかかるコスト(人的・時間的コスト)が肥大化していくという問題もあ
る。

  \section{従来のネットワークテスト}

  % 運用の理想像や現状
  % TODO: ITHD技術交流会資料 https://drive.google.com/drive/folders/0B2eRR_JxYJA5OFkzUFlveVlObWc p6-7 あたりとかの話をいれる
  
従来の「ネットワークのテスト」では、その物理構成操作の要求から、特定の場
所に人や端末を配置しながら人手でテストを実行していくという、人海戦術的な
方法がとられてきた。

しかし、ネットワーク規模や構成の大規模化・複雑化とそれによるテストパター
ン数の増大にともない、テストでパターンをすべて人手で網羅することは非常に
難しい。そのため以下のような状況(リスク)を受け入れざるをえない状況があった。
\begin{itemize}
 \item テスト作業用リソースの確保: 通常、テストをおこなうための人・機器
       の準備には制約がある。人手による作業の場合、作業コスト・時間や規
       模がどうしてもスケールさせられないため、小規模なオペレーションで
       は十分なテストができないまま本番環境での作業になる傾向がある。
 \item 一部の代表的なパターンのみをテストする: 縮小したテストケースでは
       どうしても一部の設定ミスや不整合などを見落とすリスクがある。
 \item テスト結果判断のばらつき: 手順書の解釈、操作の実行や結果の取得・
       判断などがテスト実施者に依存するため、本来問題となる事象を見落と
       してしまうリスクがある。
\end{itemize}

%%% Local Variables:
%%% mode: yatex
%%% TeX-master: main.tex
%%% End:
